\documentclass[12pt,letterpaper]{article}

%%%%% CAMBIAR LA PRIMERA LINEA POR LA SIGUIENTE PARA LA MEMORIA DE PROYECTO %%%%%
%\documentclass[12pt,letterpaper,oneside]{book}

% Paquetes basicos ...
\usepackage[utf8]{inputenc} % OJO!!!  => MANTENER ESTA LINEA PARA FACIL CONVERSION A WORD EN EL FUTURO ...
\usepackage[spanish]{babel}
\usepackage{graphicx} 
\usepackage{array}
\usepackage{tabularx}
\usepackage{amssymb, amsmath}

% Paquetes extras ... 
\usepackage{subfigure}
\usepackage{color}
\usepackage{anysize} 
\usepackage{breakcites}
\usepackage{enumitem}

\usepackage{hyperref}

\begin{document}
\marginsize{2.5cm}{2cm}{2cm}{2cm} 

% Para que no aparezca la numeracion en el pie de pagina de todo el documento ...
%\pagestyle{empty}


%%%%%% ******  INICIO CARATULA ***** %%%%%%%%%
% Especificaciones de la caratula PPG
\begin{titlepage}
\begin{center}
\vspace*{-0.5in}
\begin{large}
\textbf{UNIVERSIDAD MAYOR DE SAN ANDRÉS}\\
\vspace*{0.15in}
\textbf{FACULTAD DE INGENIERÍA}\\
\vspace*{0.15in}
\textbf{CARRERA DE INGENIERÍA ELECTRÓNICA}\\
\vspace*{0.1in}
\end{large}
% Logo UMSA
\begin{figure}[htb]
\begin{center}
\includegraphics[width=8cm]{umsa.jpg}
\end{center}
\end{figure}
\begin{Large}
\textbf{PERFIL DE PROYECTO DE GRADO} 
\end{Large}
\vspace*{0.4in}

\begin{normalsize}
\textbf{``Aprendizaje fin a fin para la conducción autónoma de vehículos domésticos usando visión artificial y redes neuronales convolucionales''} \\
\end{normalsize}



\vspace*{0.2in}


\begin{large}
\textbf{POSTULANTE:} JOSE EDUARDO LARUTA ESPEJO\\
\end{large}

\begin{large}
\hspace{0.08in} \textbf{TUTOR:} JAVIER SANABRIA GARCIA\\
\end{large}

\begin{large}
\hspace{0.44in} \textbf{D.A.M.:} GONZALO SAMUEL CABA MORALES\\
\end{large}

\vspace*{0.2in}



\begin{normalsize}
LA PAZ, AGOSTO 2018\\
\end{normalsize}
\end{center}
\end{titlepage}


\thispagestyle{empty}
% Generacion del indice
\tableofcontents
\newpage

%%%%%% ******  FIN CARATULA ***** %%%%%%%%%

% Contenido del PPG
% NOTA: presionar F11 para que se actualicen las citas bibliograficas ...

\section{Introducción}

Los vehículos autónomos han pasado de ser un tema de ciencia ficción a convertirse una realidad cada vez más 
cercana. Si bien existe un recorrido muy largo para llegar a implementar sistemas completamente autónomos en las calles, 
los recientes avances en la tecnología junto con el interés económico de grandes empresas y corporaciones en el mundo 
ha hecho posible incluir diversos niveles de autonomía a vehículos con fines de uso doméstico e industrial con éxito.

Una de las áreas que más se ha nutrido de los recientes avances es el área de la visión artificial o visión por computador; 
resolviendo con facilidad tareas de una complejidad muy alta, como la detección y reconocimiento de objetos. 
Este crecimiento, en gran parte, se ha debido al desarrollo y optimización de las redes neuronales, las cuales se han 
constituido en una herramienta con muchas potencialidades y aplicaciones por la forma en la que se procesa la información 
y su capacidad para generalizar tareas complejas en base a una gran cantidad de datos de entrenamiento. En específico, 
las redes neuronales convolucionales han podido revivir al campo de la visión artificial gracias a la forma eficiente 
en la que procesa imágenes o matricies multidimensionales y la capacidad de crear representaciones internas a partir 
de filtros de convolución.

La visión artificial juega un papel muy importante en el desarrollo de vehículos autónomos por cuanto permite 
procesar imágenes digitales provenientes de cámaras instaladas en los mismos vehículos y extraer información 
valiosa para la navegación y la conducción, como ser la detección de carril, peatones, signos de tránsito, otros vehículos, 
etc. Esta utilidad hace posible diversas oportunidades de investigación y desarrollo de algoritmos y sistemas de visión 
artificial orientados a conducción autónoma.

El presente proyecto se centra en el desarrollo de un sistema de conducción autónoma basado en visión artificial para 
la generación de comandos de control para la conducción autónoma de un vehículo doméstico con un modelo de locomoción 
de Ackerman. Se intenta desarrollar un sistema de aprendizaje “fin a fin” que consta de un modelo de predicción 
que genera comandos de control a partir de un estímulo visual proveniente de una cámara monocular.
%% detallar un poco mas el reto del sistema fin a fin

% definicion de sistemas fin a fin

% TODO: referencias 
\section{Antecedentes}


El primer intento de desarrollo de un sistema de conducción autónomo “fin a fin” fue llevado a cabo por la Agencia 
de Proyectos de Investigación Avanzada en Defensa de los Estados Unidos (DARPA) con un proyecto conocido 
como el Vehículo Autónomo de DARPA o DAVE \cite{lecun2004dave} en el cual un vehículo radio controlado a escala tenía la 
tarea de conducir a través de un entorno escabroso. El vehículo DAVE fue entrenado a partir de cientos de
horas de conducción humana en entornos similares pero no idénticos. Los datos de entrenamiento 
incluyeron imágenes de dos cámaras de video y comandos de control generados por un operador humano. 

Luego de este primer hito en la aplicación de sistemas fin a fin en tareas de conducción, 

\subsection{Sistemas de Conducción Autónoma}
Posteriormente, el primer hito en el desarrollo de un sistema completamente autónomo vino 
con la organización del DARPA “Grand Challenge” en el cual equipos de varias universidades, 
institutos de investigación y empresas tuvieron que enfrentar el difícil reto de desarrollar 
un sistema capaz de controlar un vehículo doméstico a través de una carretera 
ripiada en medio del desierto de Arizona. Dentro las 2 versiones del Darpa Grand Challenge 
destacaron los proyectos de universidades como Stanford con el robot Stanley \cite{Thrun2006} que fue el primer 
vehículo en recorrer mas de 170 kilómetros en una carretera ripiada de manera completamente autónoma. 

% IMAGEN: stanley de stanford

%%%% Figura 1 %%%%%%
\begin{figure}[!h] 
\centering
\includegraphics[width=0.5\textwidth]{stanley1}
\caption{Stanley, el vehículo autónomo de Stanford que ganó la competencia DARPA Grand Challenge en 2005. 
        Fuente: \href{http://stanford.edu/~cpiech/cs221/apps/driverlessCar.html}{stanford.edu} }
\label{fig:stanley1}
\end{figure}

El éxito de los proyectos que participaron en el grand challenge sentó un gran precedente en el desarrollo de lo que 
ahora se conoce como \textit{Self Driving Car} o vehículo autónomo. De hecho, muchos de los equipos 
participantes de este concurso se constituyen en la actualidad como existosas empresas de desarrollo o 
coadyuvan en iniciativas privadas de gigantes de la tecnología como Google, Uber o Nissan.

\subsubsection{Arquitectura general de un sistema de conducción autónoma}
En general, la arquitectura de un sistema de conducción autónoma se puede entender como la integración de varios módulos o 
subsistemas funcionales, tal como se puede observar en la figura \ref(arquitectura).


% IMAGEN: arquitectura

%%%% Figura 1 %%%%%%
\begin{figure}[!h] 
    \centering
    \includegraphics[width=0.5\textwidth]{dibujos/overview}
    \caption{Arquitectura de un sistema de conducción autónoma}
    \label{fig:arquitectura}
    \end{figure}


Sin embargo, debido al creciente interés tanto en investigación como económico en los sistemas de conducción 
autónoma, la Sociedad de Ingenieros en Automoción (SAE por sus siglas en inglés) ha elaborado un estándar donde se 
detallan distintos aspectos concernientes. La regulación define varios niveles de autonomía en 
vehículos terrestres, aéreos y acuáticos yendo desde un control completamente manual, 
normalmente observado en vehículos completamente mecánicos, pasando por asistencias al control 
hasta llegar a un vehículo completamente autónomo en todas sus tareas

% IMAGEN: niveles de autonomia
%%%% Figura 2 %%%%%%
\begin{figure}[!h] 
\centering
\includegraphics[width=0.70\textwidth]{levels}
\caption{Niveles de automatización en la conducción según SAE. 
        Fuente: \href{https://www.researchgate.net/figure/Terms-related-to-automated-driving-according-to-SAE-and-VDA_fig1_273883061}{researchgate} }
\label{fig:levels}
\end{figure}

La creación de estándares y regulación ha tenido como consecuencia que, en la actualidad, existan varias iniciativas 
en el desarrollo de los \textit{self Driving Cars}, siendo una de las más 
importantes la empresa Waymo, dependiente de Google a través de su empresa Pública Alphabet. Waymo, ha aprovechado 
el uso de tecnologías emergentes de sensado como el LIDAR para mejorar el mapeo y la navegación a través de algoritmos 
de fusión de sensores. Aparte de Alphabet, existen diversas iniciativas privadas en el desarrollo de vehículos autónomos 
con fines comerciales como los Self Driving Cars de Uber, Toyota, BMW, Ford, entre otros.

% IMAGEN: vehiculo de waymo

%%%% Figura 1 %%%%%%
\begin{figure}[!h] 
\centering
\includegraphics[width=0.5\textwidth]{waymo}
\caption{El vehículo autónomo de Waymo. 
        \href{https://waymo.com/}{waymo.com}}
\label{fig:waymo}
\end{figure}

Una de las tareas más importantes dentro de un \textit{self driving car} es la detección y mantención del carril del 
vehículo. Fabricantes de vehículos automotores han incluido con éxito sistemas de asistencia al conductor para la 
mantención del carril usando cámaras digitales y visión artificial para poder detectar la posición del automóvil con 
respecto al carril. Estos sistemas se consideran fundamentales en sistemas de conducción autónoma. Durante las últimas 
dos décadas, se han desarrollado distintos tipos de sistemas y enfoques para resolver el problema de la mantención de 
carril. En el presente proyecto, se explorará el problema desde un punto de vista de la tarea de aprendizaje supervisado
usando un sistema de aprendizaje fin a fin.

Los sistemas de aprendizaje fin a fin se han explorado de manera exitosa en los últimos años, esto debido a la creciente
disponibilidad de sistemas de cómputo de alta concurrencia, en especial las Unidades de Procesamiento Gŕafico de propósito
general o GPGPU por sus siglas en inglés. Esta disponibilidad ha logrado que se puedan entrenar redes neuronales completas 
en una estación de trabajo que no consume demasiada energía. Una de las empresas pioneras en GPGPU es Nvidia con su herramienta 
CUDA, que ha permitido el desarrollo de algoritmos de entrenamiento e inferencia para redes neuronales de manera sencilla. Es 
precisamente Nvidia que ha demostrado que los sistemas de aprendizaje fin a fin pueden tener éxito con el desarrollo de un 
prototipo y arquitectura de vehículo autónomo \cite{bojarski2016end}.
%*******************JUSTIFICACION*******************

\section{Justificación del Proyecto}

\subsection{Justificación técnica}
El proyecto se justifica desde el punto de vista técnico a partir de las técnicas y procedimientos explorados 
en el mismo para implementar un sistema de visión y control asistido basado en una red neuronal convolucional. 
Usualmente, el análisis de la aplicabilidad de una red neuronal involucra solamente la prueba sobre un dataset 
genérico donde se evalúa su rendimiento analizando el error de clasificación y pérdidas. En este caso, se plantea el 
flujo de trabajo completo orientado a la implementación en un prototipo a escala usando técnicas convencionales en 
el análisis y diseño de sistemas de visión artificial así como también la implementación del mencionado sistema en 
una plataforma de cómputo embebida.

Las tecnologías a ser usadas deberán tener la característica de pertenecer a los estándares actuales en la 
investigación e industria, esto con el fin de ser replicable y extensible en futuros proyectos.

\subsection{Justificación académica}

Desde el punto de vista académico, el proyecto se justifica en el entendido del uso de técnicas y procedimientos 
de ingeniería para el análisis y diseño de un sistema de aprendizaje “fin a fin” usando redes neuronales. Tales 
técnicas y procedimientos incluyen la definición de la arquitectura de la red neuronal, el entrenamiento y el análisis 
del rendimiento de la misma. Así como también el dimensionamiento de los componentes de cómputo embebido para el 
prototipo y la implementación de los sistemas de control electrónico de bajo nivel para el mismo. Tales técnicas y 
procedimientos se corresponden de manera integral con los conocimientos adquiridos a lo largo de la carrera 
de Ingeniería Electrónica en sus distintas asignaturas.

% \subsection{Justificación social}

% En Bolivia existe más de un millón de automotores y la gran mayoría son de uso privado \cite{INE2018}. El parque automotor va 
% en crecimiento y junto con él temas de impacto ambiental que van en desmedro de la calidad de vida de la población 
% como por ejemplo: polución del aire, contaminación sonora o mal uso de espacios públicos. En este entendido, el 
% desarrollo de vehículos autónomos surge como una oportunidad para solucionar algunos de estos problemas y mejorar 
% la calidad de vida de la población.

%******************ANALISIS DE LA PROBLEMATICA*********************
%------ TODO: completar esta parte ------------- %
\section{Análisis de la problemática y \\
planteamiento del problema}
\subsection{Análisis de la problemática}
% PROBLEMATICA: Conjunto de aspectos que causan problemas en un PG.
% ANALISIS DE LA PROBLEMATICA: Determinacion de las causas sus efectos y sus relaciones, vinculados con los aspectos problematicos de la idea del proyecto.  
% ****Es el analisis de las cuestiones discutibles que requieren de una solucion en el PG.

Se han estudiado diferentes enfoques para lograr solucionar la tarea de 
conducción autónoma para vehículos domésticos. Normalmente, la salida del 
sistema se expresa como una serie de comandos de control de aceleración y dirección 
del volante del vehículo. Estos comandos se pueden obtener de diversas maneras dependiendo 
el nivel de robustez y abstracción que el sistema requiere. Muchos sistemas 
se basan en la fusión de distintos tipos de sensores y fuentes de información como ser 
mapas satelitales, GPS, sensores láser y cámaras. La combinación de esta información 
es procesada y fusionada mediante distintos algoritmos de filtrado tales como el filtro de kalman. 
La característica de este tipo de sistema es que se puede expresar como una serie de etapas 
de procesamiento mediante el cual la información fluye y se transforma, cada una de las etapas es 
diseñada e implementada en base a conocimiento específico y con requerimientos y limitaciones específicas 
de la tarea que realiza. 

%  diagrama de sistema de conduccion autonoma completo

Si bien el enfoque anteriormente mencionado ha logrado conseguir importantes avances y resultados 
muy prometedores, involucra un gran esfuerzo a la hora de diseñar cada una de las etapas independientemente 
para luego hacer que funcionen todas juntas y cumplan la tarea asignada. Este proceso usualmente requiere 
de un equipo de expertos que sea capaz de realizar las tareas de diseño de las etapas o módulos del sistema 
y el de la integración de los módulos en un solo sistema funcional. Este enfoque, pese a que ha demostrado ser 
una forma efectiva de trabajo para diversos problemas, tiene la desventaja de requerir muchos 
recursos y tiempo para poder lograr un sistema funcional. 


%
\subsection{Planteamiento del problema}
% PLANTEAMIENTO DEL PROBLEMA: Tiene que ver con enfocar la solucion de los aspectos problematicos tratados, 
% desde su situacion inicial hasta su situacion final deseada.

% definir el concepto de informacion con referencia a datos
% 

Considerando a la tarea de conducción autónoma como un sistema de procesamiento de datos, se debe tener en 
cuenta varios aspectos concernientes tanto al diseño como a la implementación de dichos tipos de sistemas. Una 
de las principales características es que tradicionalmente el flujo de la información pasa por varias etapas.

En el área de visión por computadora para tareas de conducción autónoma, normalmente se sigue el siguiente flujo en el desarrollo 
un sistema o prototipo:

\begin{enumerate}
    \item \textbf{Extracción de características.} Esta etapa incluye el preprocesamiento y transformación de la imagen 
    en un conjunto de características de distinta índole. Estas características se suelen llamar también descriptores 
    y sirven para describir los aspectos más relevantes de la imagen para la tarea final, por ejemplo, la detección de bordes. 
    La extracción de características también se usa para reducir la dimensionalidad inicial de la imagen a una más tratable y 
    amigable con la capacidad de procesamiento computacional disponible. Las características o descriptores a usarse 
    se definen manualmente por medio de conocimiento experto y se afinan de la misma manera.

    \item \textbf{Algoritmo de predicción.} Esta etapa incluye típicamente un algoritmo de aprendizaje previamente 
    entrenado con un conjunto de datos adecuado, permite realizar distintas tareas de alto nivel sobre los descriptores 
    obtenidos de la imagen. Estas descripciones de alto nivel incluyen normalmente tareas de detección, 
    clasificación o regresión. Los algoritmos de aprendizaje incluyen típicamente algoritmos básicos, tales como árboles de decisión,
    regresión lineal o máquinas de soporte vectorial ya que deben realizar la tarea de predicción en un 
    conjunto de dimensionalidad relativamente baja (los descriptores).

    \item \textbf{Adecuación de los datos de salida.} La información extraída de la anterior etapa debe 
    procesarse para poder ser traducida a comandos de control que actuen directamente con las etapas 
    de bajo nivel del vehículo, es decir la etapa de actuación y potencia. En esta etapa se suele incluir algún algoritmo 
    de control realimentado para el control de motores así como también algoritmos de fusión de distintas fuentes de información
    para obtener finalmente una señal de comando para los actuadores.
\end{enumerate}

Como se ha podido observar, el flujo de trabajo en un sistema de conducción autónomo se compone de varias etapas 
interdependientes que se deben realizar con conocimiento y experiencia específica en cada una de las mismas. 

Por su parte, otra de las dificultades con este acercamiento al reto de la conducción autónoma es el de la reducida
flexibilidad del sistema. En otras palabras, si se quisiera modificar el sistema para agregar requerimientos o 
expandir la funcionalidad del mismo, se debe realizar una modificación a la etapa específica y evaluar el impacto de 
las modificaciones en todo el sistema en su conjunto. Esto dificulta de manera sustancial la reutilización de diversos
componentes en sistemas similares.

Finalmente, la exagerada complejidad y conocimientos requeridos para poder implementar un sistema experimental 
de esta naturaleza hace prácticamente imposible su desarrollo por equipos de investigación pequeños o investigadores 
individuales. Dada la importancia y la potencialidad de los sistemas de conducción autónoma es escencial reducir 
esta dificultad de implementación y experimentación.

%------------------
\section{Objetivos}
\subsection{Objetivo General}

Diseñar un sistema de aprendizaje “fin a fin” para la generación de comandos de 
dirección en la tarea de conducción autónoma de vehículos domésticos basado en 
visión artificial y redes neuronales convolucionales.

\subsection{Objetivos Específicos}
Para alcanzar el objetivo general será necesario:

\begin{itemize}
    \item Investigar y evaluar requerimientos específicos para el sistema.
    \item Diseñar una plataforma de prueba en hardware con características similares a las de un vehículo doméstico real.
    \item Diseñar la arquitectura de una red neuronal capaz de cumplir con la tarea de generación de comandos 
    de control para un vehículo autónomo.
    \item Realizar pruebas de rendimiento y análisis comparativos en base al sistema implementado.
    \item Proponer un prototipo funcional en el que se pueda apreciar los alcances del sistema.
\end{itemize}


\section{Alcance}

Dentro de los alcances del proyecto se puedencitar los siguientes
aspectos:

\begin{itemize}
    \item Recopilación de información de distintas fuentes en 
    en trabajos de investigación.
    \item El desarrollo de un prototipo funcional a escala con 
    locomoción de Ackerman.
    \item El diseño electrónico del circuito de control y comunicación de
    la Computadora de Abordo.
    \item Investigación de arquitecturas y plataformas de software para el
    diseño y despliegue de robots móviles y tareas de conducción autónoma.
    \item El diseño de la arquitectura de la red neuronal en base a 
    especificaciones de funcionamiento, desempeño y plataforma de hardware.
    \item Uso de algoritmos de preprocesamiento de datos y funciones
    de agregación de datos previos al entrenamiento de la red neuronal.
    \item El entrenamiento de la red neuronal en un entorno externo a la OBC.
    \item El análisis de los errores de entrenamiento y validación de la red neuronal.

\end{itemize}

\section{Límites}
El sistema, por su parte, contará con ciertas restricciones detalladas a 
continuación:
\begin{itemize}
    \item En la recopilación de información y fuentes se 
    considerará principalmente investigaciones y trabajos relacionados
    con tareas de aprendizaje "fin a fin" y redes neuronales convolucionales.
    \item El prototipo a escala servirá solamente para un análisis superficial 
    de la dinámica de un vehículo automotor doméstico tomando como punto 
    de inicio modelos matemáticos simplificados y limitaciones de rangos de trabajo dentro 
    de dichos modelos.
    \item la electrónica de los sistemas de control y percepción estará 
    diseñada específicamente para el modelo a escala y las limitaciones anteriormente
    mencionadas.
    \item Se explorarán exclusivamente plataformas, librerías y herramientas
    de cógo abierto y con licencias de uso no privativas.
    \item El diseño de la arquitectura de la red neuronal estará orientado a 
    tareas de aprendizaje supervisado y aproximación de funciones.
    \item tanto la arquitectura de la red como los hiperparámetros de 
    la misma estarán acotados considerando las limitaciones de memoria y capacidad 
    de procesamiento ofrecidos por la plataforma de cómputo o Computadora
    de Abordo disponible.
    \item La recolección de datos estará enfocada y definida por el prototipo
    a implementar.
\end{itemize}

\section{Solución Propuesta}

La propuesta del presente proyecto consta del desarrollo de un sistema de aprendizaje
fin a fin para la conducción autónoma de vehículos automotores domésticos. En este
sentido, se puede desglosar la arquitectura del sistema propuesto dividido en 
distintos subsistemas funcionales.
Por su parte, también es importante mencionar que el sistema en su conjunto contará 
con tres subsistemas intedependientes: Subsistema de adquisición de datos y 
entrenamiento, subsistema de inferencia y conducción autónoma, subsistema de control 
y actuación.

\subsection{Subsistema de adquisición de datos y entrenamiento.}
Este subsistema se compone de un conjunto de herramientas y utilidades para la 
adquisición de un nuevo conjunto de datos de entrenamiento y validación, así como
también un módulo de entrenamiento de la red neuronal convolucional para la tarea de
la generación de comandos de control.

\subsection{Subsistema de inferencia y conducción autónoma.}
El subsistema de inferencia y conducción autónoma tiene la tarea de obtener y ejecutar
las predicciones de la red neuronal entrenada en el Sistema de Adquisición de Datos y Entrenamiento.
Este subsistema tomará como entradas los datos de imágenes de una cámara y datos de 
sensores a bordo del prototipo para generar comandos de control acordes con el entorno
percibido. Este subsistema estará implementado en forma de un programa de ``piloto automático''
capaz de conducir el prototipo de manera autónoma.

\subsection{Subsistema de control y actuación.}
Este subsistema compone la parte física de la propuesta. Contará con un modelo físico
a escala con locomoción de Ackerman con actuadores para la tracción y la dirección. 
Además, también contará con una plataforma electrónica embebida que se conectará 
directamente a los actuadores a través de una etapa de potencia y a los sensores 
correspondientes mediante un sistema de tiempo real. La parte lógica estará implementada
en una computadora de abordo que servirá de puente entre el mundo físico y la parte 
lógica de todo el sistema.



% %%%% Figura 1 %%%%%%
% \begin{figure}[!h] 
% \centering
% \includegraphics[width=0.5\textwidth]{esquema1}
% \caption{La leyenda viene aca ...}
% \label{fig:esquema1}
% \end{figure}


% %%%% Figura 2 %%%%%%
% \begin{figure}[!h] 
% \centering
% \includegraphics[width=0.5\textwidth]{esquema2}
% \caption{La leyenda viene aca ...}
% \label{fig:esquema2}
% \end{figure}ch

\section{Temario del Proyecto}
A continuación se presenta el temario tentativo de la memoria del proyecto, 
considerando los posibles contenidos, sin un detalle exhaustivo de los mismos 
puesto que podría ser una limitante en la estructura final. 
Inicialmente se considera la siguiente estructura:\\

\begin{itemize}

\item \textbf{Título.}

\item \textbf{Resumen.}

\item \textbf{Dedicatoria.}

\item \textbf{Agradecimientos.}

\item \textbf{Lista de Figuras.}

\item \textbf{Capitulo 1: Introducción}  

\item \textbf{Capitulo 2: Fundamentos del proyecto}  

\item \textbf{Capitulo 3: Marco práctico del proyecto} 

\item \textbf{Capitulo 4: Análisis y discusión de resultados}

\item \textbf{Capitulo 5: Conclusiones y recomendaciones}

\item \textbf{Referencias y bibliografía}

\item \textbf{Glosario de términos}

\item \textbf{Anexos}
%\renewcommand{\theenumi}{\thesection.\arabic{enumi}}
\end{itemize}

\newpage
\section{Cronograma del Proyecto}

%%%% Figura 1 %%%%%%
\begin{figure}[!ht] 
    \centering
    \includegraphics[width=0.85\textwidth]{cronograma}
    \caption{Diagrama de Gantt para la elaboración del proyecto de grado. Fuente: Elaboración propia}
    \label{fig:cronograma}
    \end{figure}

\newpage
\section{Bibliografia}

\bibliographystyle{IEEEtran}
\bibliography{referencias}
\end{document}
